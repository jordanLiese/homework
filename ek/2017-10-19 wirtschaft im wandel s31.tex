\documentclass[a4paper,12pt]{article}

\usepackage[top = 2.5cm, bottom = 2.5cm, left = 2.5cm, right = 2.5cm]{geometry} 

\usepackage{booktabs}

\usepackage[ngerman]{babel}
\usepackage{float}
\usepackage{fancyhdr}

\def \dtitle {Erdkunde: Wirtschaft im Wandel}


\pagestyle{fancy}

\fancyhf{}

\lhead{\footnotesize \dtitle}

\rhead{\footnotesize 3}

\cfoot{\footnotesize \thepage} 

\begin{document}
\thispagestyle{empty}

\begin{tabular}{p{15.5cm}}
{\large \bf {\dtitle}} \\
IGS OS\\ Oktober 2017 \hspace*{10.5cm} Jordan Liese\\
\midrule
\\
\end{tabular}

\vspace*{0.3cm}
\vspace{0.4cm}


\begin{enumerate}
\item{\textit{Beschrieben Sie die Zyklen der Wirtschaftsentwicklung nach Kondratjew.}}
\end{enumerate}

Material \emph{M6} zeigt die Zyklen der Wirschaftentwicklung nach Kondratjew, welcher die wirtschaftliche Aktivität in mehrere Zyklen unterteilt.
Jeder Zyklus besteht aus 3 Phasen:
\begin{description}
    \item [dem Aufschwung] Tief greifende Erfindungen und Innovationen bieten neue wirtschafliche M\"oglichkeiten, welche die wirtschaftliche Aktivit\"at steigern.
        Es etablieren sich \textit{dominante Branchen}, welche die neuen Technologien und Resuorcen nutzen und das Handelsvolumen steigt an.
    \item [dem Abschwung] Nachdem sich die Wirschaft an die neuen M\"oglichkeiten angepasst hat, stagniert sie und die wirtschaftliche Aktivit\"at beginnt zu sinken.
        Durch die Rezession treten Probleme auf und die Bedingungen f\"ur Unternehmen sind ung\"unstig.
    \item [dem Tiefpunkt] die wirtschaftliche Aktivit\"at ist an einem Tiefpunkt und es treten Probleme auf; die Motivation zur Ver\"anderung steigt und es folgt eine neue
        Phase des Aufschwungs.
\end{description}
Ein Zyklus dauert um die 50 Jahre und wird von einem Zyklus gefolgt, welcher einen gr\"o{\ss}eren Hochpunkt und niedriegeren Tiefpunkt besitzt.

\begin{enumerate}
    \item{\textit{Charakterisieren Sie den intersektoralen Strukturwandel der deutschen Industrie im Kontext von M6.}}
\end{enumerate}
Intrasektoraler Strukturwandel bezeichnet eine Ver\"anderung der Produktionspalette und Organisationsstruktur innerhalb eines Sektors, wie es auch in \textit{M8} passiert: Es wird sich auf die Produktion
techonologieintensiver Produkte konzentriet; es werden mehr hochqualifizierte Arbeiter gebraucht um die neuen Technologien zu nutzen.
\end{document}
