\documentclass[a4paper, 12pt]{article}

\usepackage{booktabs}

\usepackage[ngerman]{babel}
\usepackage{float}
\usepackage{fancyhdr}
\usepackage{amsmath}

\def \dtitle {Chemie: Auswirkungen von Alkohol}

\pagestyle{fancy}
\fancyhf{}
\lhead{\footnotesize \dtitle}
\rhead{\footnotesize 3}
\cfoot{\footnotesize \thepage} 

\begin{document}
\thispagestyle{empty}

\begin{tabular}{p{15.5cm}}
{\large \bf \dtitle} \\
IGS OS\\ September 2017 \hspace*{10.5cm} Jordan Liese\\
\midrule
\\
\end{tabular}

\begin{enumerate}
    \item Aufgabe A1
\end{enumerate} 

Alkohol st\"ort die Funktion von Nervenzellen und senkt dadurch Denkverm\"ogen oder Selbstkontrolle.
Auch werden Hemmung gemindert, sodass Konsumenten von Alkohol unvorsichtiger fahren.
Man bewegt sich zudem langsamer und Bewegungen werden unkontrolliert. Zudem engt Alkoholkonsum das Sichtfeld ein.

\begin{enumerate}
    \item Aufgabe A2
\end{enumerate} 

0.5 Promille oder 0.25 mg/l sind ordnungswriedrig. (§ 24a 1 StVG dejure.org) 
Fuer das senken der Promillegrenze spricht, dass nach Einfuehrung der 0.0 Promillegrenze für Fahranfaenger
eingeführt wurde Unfallraten sanken. (Bsp. Brandenburg 12 %)

\begin{enumerate}
    \item Aufgabe A3
\end{enumerate} 

Der Blutalkoholgehalt \[w\] einer 50kg schweren Frau ist zu berechnen. Sie hat 2 (zwei) Flaschen Alcopops zu sich genommen. Jede Flasche enthält 0,33l; \[ \sigma = 0.5\]

\begin{equation}
\begin{split}
w & = \frac{m(Alkohol)}{m(Person) * r}\\
V(Alkohol) & = 2 * 0.33 * \sigma = 0.025\\
r & = 0.6\\
m(Person) & = 50 kg\\
m(Alkohol) & = 0.953 \permil\\
\end{split}
\end{equation}
\end{document}
    